\documentclass{exam}
\usepackage[utf8]{inputenc}
\usepackage[version=4]{mhchem}
\usepackage{chemfig}
\usepackage[margin = 1in]{geometry}
\usepackage{multicol}
\setlength{\columnsep}{6cm}


\begin{document}
\noindent\makebox[\textwidth]{Name and Homeroom:\enspace\hrulefill}

\noindent \textbf{Identify the following compounds, using standard IUPAC rules.}
\begin{multicols}{2}
\begin{questions}
\question \ce{[Mo(CN)_8]^{-4}} \boxed{\text{Octacyanomolybdenate (IV) anion}}
\question \ce{[PtCl_3(C_2H_4)]^-} \boxed{\text{Trichloroethylplatinate (II) anion}}
\question \chemfig{
             H_3C% 1
    -[:60,,2]S% 2
                (
          =[:150]O% 4
                )
                (
       -[:60,,,1]CH_3% 5
                )
      =[:330]O% 3
} \boxed{\text{Dimethyl sulfone OR Methylsulfonylmethyl (MSM)}}
\question \chemfig{
               % 1
        -[:270]% 2
                  (
            =[:330]O% 3
                  )
    -[:210,,,2]HO% 4
} \boxed{\text{Formic acid OR Methanoic acid}}
\question \chemfig{
                CH_2% 1
    =[:180,,1,2]H_2C% 2
} \boxed{\text{Ethene}}
\question \chemfig{
           F% 2
    -[:240]% 1
              (
        -[:150]F% 3
              )
              (
        -[:240]F% 5
              )
    -[:330]F% 4
} \boxed{\text{Tetrafluoromethane OR Carbon tetrafluoride}}
\question \chemfig{
           % 1
     -[:30]% 2
    -[:330]O% 3
     -[:30]% 4
    -[:330]% 5
} \boxed{\text{Diethyl ether OR Ethoxyethane}}
\question \chemfig{
               CH_3% 1
     -[:210,,1]% 2
        -[:150]% 3
        -[:210]% 4
        -[:150]% 5
        -[:210]% 6
        -[:150]% 7
        -[:210]% 8
                  (
            =[:150]O% 9
                  )
    -[:270,,,1]OH% 10
} \boxed{\text{Octanoic acid}}
\question \chemfig{
               HO% 8
     -[:300,,2]% 6
              -% 5
                  (
        <:[:60,,,1]OH% 9
                  )
        -[:300]% 4
                  (
            <[,,,1]OH% 10
                  )
        -[:240]% 3
                  (
       <:[:300,,,1]OH% 11
                  )
        -[:180]% 2
                  (
            -[:120]O% 7
             -[:60]% -> 6
                  )
        <[:240]% 1
    -[:180,,,2]HO% 12
} \boxed{\text{Glucose}}
\question \chemfig{
              OH% 1
    -[:270,,1]Br% 2
                 (
           =[:330]O% 3
                 )
       =[:210]O% 4
} \boxed{\text{Bromic acid}}
\question \chemfig{
               O% 4
        =[:270]% 3
                  (
        -[:330,,,1]OH% 5
                  )
        -[:210]% 2
                  (
        -[:270,,,1]OH% 6
                  )
        -[:150]% 1
                  (
         -[:90,,,1]OH% 10
                  )
        -[:210]% 7
                  (
            =[:270]O% 8
                  )
    -[:150,,,2]HO% 9
} \boxed{\text{Tartaric acid OR Dihydroxybutanedioic acid}}
\question \chemfig{H-C(-[2]H)(-[6]H)-C(-[7]H)=[1]O} \boxed{\text{Ethanal}}
\question \chemfig{
           % 1
     -[:90]S% 2
              (
         -[:30]% 4
              )
    =[:150]O% 3
} \boxed{\text{Dimethyl sulfoxide}}
\question \chemfig{*6(=-=-=-)} \boxed{\text{Benzene}}
\question \chemfig{
              O% 3
        =[:30]% 2
                 (
        -[:90,,,1]OH% 4
                 )
       -[:330]% 1
    -[:30,,,1]NH_2% 5
} \boxed{\text{Aminoacetic acid OR glycine}}
\question \chemfig{-[:30]=[:-30]-[:30]} \boxed{\text{Trans-butene}}
\question \chemfig{
              H_2C% 1
      -[,,2,1]CH_2% 2
    -[:120,,1]O% 3
                 (
           -[:240]\phantom{C}% -> 1
                 )
} \boxed{\text{Ethylene oxide}}
\question \chemfig{
           % 1
    -[:180]% 2
    -[:240]% 3
    -[:300]% 4
          -% 5
     -[:60]% 6
              (
        -[:120]% -> 1
              )
} \boxed{\text{Cyclohexane}}
\question \chemfig{
              H_3C% 1
     -[:30,,2]S% 2
       -[:330]S% 3
    -[:30,,,1]CH_3% 4
} \boxed{\text{Dimethyl disulfide}}
\question \chemfig{
              PH_2% 7
    -[:180,,1]% 4
      =_[:240]% 3
       -[:180]% 2
      =_[:120]% 1
        -[:60]% 6
            =_% 5
                 (
           -[:300]% -> 4
                 )
} \boxed{\text{Phenylphosphene}}


\end{questions}

\end{multicols}

\end{document}